% !Rnw root = ../article.Rnw

\section[SKG tries an Intro]{SKG Intro} \label{sec:skg_intro}

As scientific resources have increased through the years, the number of models to analyze a single phenomenon has also increased.  This is evidenced by the appearance of dozens of publicly available estimates of the reproduction number $R_0$ for COVID-19 \citep{}.  As both the quantity of data and models increase, the `compare and assess' step of the data analysis pipeline of infectious disease also increases in importance.

In Figure \ref{fig:pipeline}, we illustrate the data analysis pipeline of infectious diseases as a) data pre-processing, b) exploratory data analysis (EDA), c) modeling and simulating, d) post-processing, and e) comparison and assessment, where each previous part influences the next.  We introduce \pkg{EpiCompare}, a \texttt{R} package used to complement and enhance the available tools for every part of the pipeline, with special emphasis on tools a) to harmonize input and output data and b) to compare multiple infectious disease data sets or infectious disease models to one another.

The goal of \pkg{EpiCompare} is not to supplant existing infectious disease modeling tools and software but, rather, is a concerted effort to create \textit{fair} comparisons among models developed for outbreaks and outbreak data.  To reach this goal, \pkg{EpiCompare} 

\begin{itemize}
  \item adds \textbf{harmonization} tools for a) standardizing outputs from common infectious disease model software  and b) converting individual/agent level data into a format that is comparable to aggregate models
\end{itemize}


The paper is arranged as follows.  



