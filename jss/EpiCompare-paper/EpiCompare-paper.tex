% To remove when JSS.cls is fixed
% https://github.com/rstudio/rticles/issues/329
\RequirePackage[2020/02/02]{latexrelease}
\documentclass[
]{jss}

\usepackage[utf8]{inputenc}

\providecommand{\tightlist}{%
  \setlength{\itemsep}{0pt}\setlength{\parskip}{0pt}}

\author{
Shannon K. Gallagher\\Biostatistics Research Branch\\
National Institute of Allergy\\
and Infectious Diseases \And Benjamin Leroy\\Dept. of Statistics\\
Carnegie Mellon University
}
\title{Time invariant analysis of epidemics with \pkg{EpiCompare}}

\Plainauthor{Shannon K. Gallagher, Benjamin Leroy}
\Plaintitle{A Capitalized Title: Something about a Package foo}
\Shorttitle{\pkg{foo}: A Capitalized Title}

\Abstract{
The abstract of the article.
}

\Keywords{keywords, not capitalized, \proglang{Java}}
\Plainkeywords{keywords, not capitalized, Java}

%% publication information
%% \Volume{50}
%% \Issue{9}
%% \Month{June}
%% \Year{2012}
%% \Submitdate{}
%% \Acceptdate{2012-06-04}

\Address{
    Shannon K. Gallagher\\
    Biostatistics Research Branch\\
National Institute of Allergy\\
and Infectious Diseases\\
    5603 Fishers Lane\\
Rockville, MD 20852\\
  E-mail: \email{shannon.gallagher@nih.gov}\\
  URL: \url{http://skgallagher.github.io}\\~\\
      Benjamin Leroy\\
    Dept. of Statistics\\
Carnegie Mellon University\\
    5000 Forbes Ave.\\
Pittsburgh, PA 15213\\
  E-mail: \email{bpleroy@andrew.cmu.edu}\\
  URL: \url{https://benjaminleroy.github.io/}\\~\\
  }

% Pandoc citation processing

% Pandoc header
\usepackage{booktabs}
\usepackage{longtable}
\usepackage{array}
\usepackage{multirow}
\usepackage{wrapfig}
\usepackage{float}

\usepackage{amsmath}

\begin{document}

\section{Introduction}\label{sec:intro}

This is the Section \ref{sec:intro}. This template demonstrates some of
the basic LaTeX that you need to know to create a JSS article.

\hypertarget{code-formatting}{%
\subsection{Code formatting}\label{code-formatting}}

In general, don't use Markdown, but use the more precise LaTeX commands
instead:

\begin{itemize}
\item
  \proglang{Java}
\item
  \pkg{plyr}
\end{itemize}

One exception is inline code, which can be written inside a pair of
backticks (i.e., using the Markdown syntax).

If you want to use LaTeX commands in headers, you need to provide a
\texttt{short-title} attribute. You can also provide a custom identifier
if necessary. See the header of Section \ref{r-code} for example.

\section[R code]{\proglang{R} code}\label{r-code}

Can be inserted in regular R markdown blocks.

hags hags hags \cite{Neal2004}

\begin{CodeChunk}
\begin{CodeInput}
R> x <- 1:10
R> x
\end{CodeInput}
\begin{CodeOutput}
 [1]  1  2  3  4  5  6  7  8  9 10
\end{CodeOutput}
\end{CodeChunk}

\subsection[Features specific to rticles]{Features specific to
\pkg{rticles}}\label{features-specific-to}

\begin{itemize}
\tightlist
\item
  Adding short titles to section headers is a feature specific to
  \pkg{rticles} (implemented via a Pandoc Lua filter). This feature is
  currently not supported by Pandoc and we will update this template if
  \href{https://github.com/jgm/pandoc/issues/4409}{it is officially
  supported in the future}.
\item
  Using the \texttt{\textbackslash{}AND} syntax in the \texttt{author}
  field to add authors on a new line. This is a specific to the
  \texttt{rticles::jss\_article} format.
\end{itemize}

In this section, we highlight a number of the functionalities available
in \proglang{EpiCompare}. These functionalities include data cleaning,
visualization, simulation, and comparison, in accordance with the data
analysis pipeline REF\ref{}. We show a full data analysis from beginning
to end that can be accomplished in a streamlined and standardized
manner.

\hypertarget{data-and-eda}{%
\subsection{Data and EDA}\label{data-and-eda}}

We analyze an outbreak of measles in the town of Hagelloch, Germany from
1861-1862, a data set organized by \cite{pfeilsticker1863}. The data was
later made visible by \cite{oesterle1992} and made available in an
\proglang{R} by \cite{surveillance2017}. The Hagelloch data includes a
rich set of features including household members, school level,
household locations, date of first symptoms (prodromes), date of measles
rash, and even the alleged infector. Because of these rich features,
this data set has been an ideal testing ground methodology in infectious
disease epidemiology and is used in work by
\cite{Neal2004,britton2011,groendyke2012,becker2016}.

\begin{CodeChunk}
\begin{CodeInput}
R> devtools::load_all()
R> library(ggplot2)
R> library(tidyr)
R> library(dplyr)
R> library(knitr)
R> library(kableExtra)
\end{CodeInput}
\end{CodeChunk}

\begin{CodeChunk}
\begin{CodeInput}
R> hagelloch_raw %>% select(PN, NAME, AGE, SEX, HN, PRO, ERU, IFTO) %>%
+   head(5) %>% kable(format = "latex", booktabs = TRUE, caption = "Cool table") %>%
+   kable_styling(position = "center")
\end{CodeInput}
\begin{table}

\caption{\label{tab:hagelloch-subset-view}Cool table}
\centering
\begin{tabular}[t]{rlrlrllr}
\toprule
PN & NAME & AGE & SEX & HN & PRO & ERU & IFTO\\
\midrule
1 & Mueller & 7 & female & 61 & 1861-11-21 & 1861-11-25 & 45\\
2 & Mueller & 6 & female & 61 & 1861-11-23 & 1861-11-27 & 45\\
3 & Mueller & 4 & female & 61 & 1861-11-28 & 1861-12-02 & 172\\
4 & Seibold & 13 & male & 62 & 1861-11-27 & 1861-11-28 & 180\\
5 & Motzer & 8 & female & 63 & 1861-11-22 & 1861-11-27 & 45\\
\bottomrule
\end{tabular}
\end{table}

\end{CodeChunk}

\begin{itemize}
\item thing
\end{itemize}

\bibliography{EpiCompare.bib}


\end{document}
